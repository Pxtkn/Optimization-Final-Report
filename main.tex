% !Mode:: "TeX:UTF-8"% !TEX TS-program = xelatex
% !TEX encoding = UTF-8 Unicode
% !Mode:: "TeX:UTF-8"

%+++++++++++++++++++++++++++++++++++++++++++++++++++++++++++++++++++++++++++++
% This is a sample article script. All rights reserved.
% Author: qianhui@zju.edu.cn
%+++++++++++++++++++++++++++++++++++++++++++++++++++++++++++++++++++++++++++++
\documentclass[a4paper,twoside,AutoFakeBold]{article}
\usepackage{optreport}

%+++++++++++++++++++++++++++++++++++++++++++++++++++++++++++++++++++++++++++++
% Some packages for this sample.
%+++++++++++++++++++++++++++++++++++++++++++++++++++++++++++++++++++++++++++++
\usepackage{comment}	% Package for comment useless document
\usepackage{bm}			% Package for Bold-math symbol
\usepackage{mathrsfs}	% Package for RSFS fonts in maths
\usepackage{listings}	% Package for Listing code
\usepackage{enumerate}	% Package for enumerate
\usepackage{pdfrender}
\usepackage{subcaption}
\usepackage{multirow}
\usepackage{booktabs}
\usepackage{bbm}
\usepackage{bbding}
\usepackage{mathtools}
\usepackage{graphicx}
\usepackage{float}
\usepackage{epstopdf}
\usepackage{lipsum}
\usepackage{metalogo}

%+++++++++++++++++++++++++++++++++++++++++++++++++++++++++++++++++++++++++++++
% Title, Authors, Reprot Time.
%+++++++++++++++++++++++++++++++++++++++++++++++++++++++++++++++++++++++++++++
\serialnum{2024-3-1145141919}

\rptname{Optimal Transport Based Distributed \gaplongcap Optimization Research}

\rptauthora{郑皓壬}{3220103230} %作者1和学号
\rptauthorb{郑俊达}{3220103540} %作者2和学号
\rptauthorc{李瀚轩}{3220106039} %作者3和学号
\reporttime{2024}{1}

% -------------------------------------------------
% for english version.
% -------------------------------------------------
\rptcontentsname{Contents}
\renewcommand{\abstractname}{{\xiaosan Abstract}}
\def\bibetal{et al.}
\def\biband{and}
\makeatletter
\renewcommand*{\ALG@name}{{\xiaosi Algorithm.~}}
\makeatother
\theoremstyle{definition}
\newtheorem{defn2}{{Definition}}
\newtheorem{corr2}{{Corrollary}}
\newtheorem{thrm2}{{Theorem}}
\newtheorem{lema2}{{Lemma}}
\newtheorem{exmp2}{{Example}}
\newtheorem{remark2}{{Remark}}
\newtheorem{assumption}{Assumption}[section]
\renewcommand*{\proofname}{{\heiti Proof.~}}
\renewcommand{\figurename}{Fig.~}
\renewcommand{\tablename}{Tab.~}
\renewcommand{\refname}{Reference}
% -------------------------------------------------

%+++++++++++++++++++++++++++++++++++++++++++++++++++++++++++++++++++++++++++++
% Document.
%+++++++++++++++++++++++++++++++++++++++++++++++++++++++++++++++++++++++++++++
\begin{document}
\pagenumbering{gobble}

%-----------------------------------------------------------------------------
%  Title Page
%-----------------------------------------------------------------------------
\maketitle
\thispagestyle{empty} \cleardoublepage

%-----------------------------------------------------------------------------
%  Table of Content
%-----------------------------------------------------------------------------
\rptcontent \thispagestyle{empty} \cleardoublepage

%-----------------------------------------------------------------------------
%  Abstract
%-----------------------------------------------------------------------------
\begin{abstract}\kaiti \xiaosi
% 论文的摘要,大致翻译一下原论文的摘要即可
\end{abstract}
\cleardoublepage

%-----------------------------------------------------------------------------
%  Sections
%-----------------------------------------------------------------------------
\pagenumbering{arabic}\songti\xiaosi
%-----------------------
%
%-----------------------
\section{论文介绍}
\subsection{问题背景}
% 论文的问题背景,对应原论文 Introduction 中 Conrtribution 之前的部分,应该比较简单
\cite{zhangzhihua2021quasinewton1}

\subsection{论文贡献}
% 论文的贡献,对应原论文 Introduction 中的 Conrtribution 部分,应该比较简单

\subsection{章节组织}
% 论文的章节组织,无对应部分,大概是要简单介绍一下原论文的章节组织?

%-----------------------
%
%-----------------------
\section{相关工作}\label{section:related}
% 论文的相关工作,对应原论文 Introduction 中的 Related Work 部分,应该比较简单

%-----------------------
%
%-----------------------
\section{问题描述和常用记号}\label{section:preliminary}
% 对应原论文的 Preliminary 一节,需要介绍经典 BFGS 和 Greedy-BFGS 算法
% 里面的引理的证明应该不用写吧?可以加一句“证明请详见原论文附录”?

\subsection{BFGS算子与算法}

\subsection{Greedy-BFGS算法}

%-----------------------
%
%-----------------------
\section{方法描述}\label{section:methods}
% 对应原论文的 Sharpened-BFGS 一节,需要介绍 Sharpened-BFGS 算法的思路及其用到的各种引理、定理
% 应该也不用证明吧?不过加上证明可以水不少分(x)
% 我觉得原论文中有一些没详细说的部分可以扩展一下,比如关于 Newton's Decrement 的定义,可以写一下 $\lambda$ 的公式是怎么来的

Sharpened-BFGS算法的核心思想就是同时使用经典的BFGS算法和贪心BFGS算法来更新$G_t$。经典的BFSG算法可以提升牛顿方向的近似估计,而贪心BFGS算法能够提高整体Heissan矩阵的估计精确度。

\subsection{二次规划}

对于一般的优化问题$\mathop{\min}\limits_{x\in\mathbb{R}^n}f(x)$,我们考虑目标函数为二次函数的情形,即
\begin{equation}
    \mathop{\min}\limits_{x\in\mathbb{R}^d}f(x)=\frac{1}{2}x^TAx+b^Tx
\end{equation}

其中$A\in\mathbb{R}^{d\times d}$是对称正定矩阵,满足条件$\mu I\preceq A\preceq LI$, $b\in\mathbb{R}^d$。

对于这类优化问题,Sharpened-BFGS算法在每轮迭代的时候进行了两次更新。算法首先通过经典BFGS算法更新得到矩阵$\overline{G}_t$,然后使用贪心BFGS算法选择下降方向$\overline{u}$, 最后由$\overline{G}_t$和$\overline{u}$更新得到$G_{t+1}$。
即Shapened-BFGS通过经典BFGS方向对贪心BFGS得到的Hessian矩阵估计进行了改进。伪代码如下:

\begin{algorithm}[H]\xiaosi
    \caption{\xiaosi Sharpened-BFGS applied to quadratic programming}
    \begin{algorithmic}
        \REQUIRE 初始化变量$x_0$, 初始化近似矩阵$G_0=LI$
        \FOR{$t=0,1,2,\cdots$}
            \STATE 更新变量 $x_{t+1}=x_t-G_t^{-1}\nabla f(x_t)$;
            \STATE 计算BFGS方向 $s_t=x_{t+1}-x_t$;
            \STATE 计算BFGS矩阵 $\overline{G}_t=BFGS(A,G_t,s_t)$;
            \STATE 计算贪心BFGS方向 $\overline{u}=\overline{u} (A,\overline{G}_t)$;
            \STATE 估计Hessian矩阵 $G_{t+1}=BFGS(A,\overline{G}_t,\overline{u}_t)$;
        \ENDFOR
    \end{algorithmic}
\end{algorithm}

为了定量地描述Sharpened-BFGS算法的收敛速度,验证其确实融合了经典BFGS和贪心BFGS算法的优点,我们首先需要定义牛顿衰减因子:
\begin{equation}
    \lambda_f (x)=\sqrt{\nabla f(x)^T\nabla ^2 f(x)^{-1}\nabla f(x)}
\end{equation}
其是用于衡量优化问题迭代过程中每一步近似解的改进程度的指标。它比较当前点的目标函数值和通过二阶信息进行近似预测的下降量。
对于其形式的推导如下:

我们想要最小化$f(\bm{x})$,考虑它的二阶泰勒展开:
\begin{equation}
    f(\bm{x})\approx f(\bm{x_0})+(\bm{x}-\bm{x_0})^T\nabla f(\bm{x_0})+\frac{1}{2}(\bm{x}-\bm{x_0})^T\nabla^2 f(\bm{x})(\bm{x}-\bm{x_0})
\end{equation}
由于海森矩阵$\nabla^2 f(\bm{x_0})$是正定的,所以为了使$f(\bm{x})$最小,我们将上式对$\bm{x}$求偏导,得到:
\begin{equation}
        \nabla f(\bm{x_0})+\nabla ^2 f(\bm{x_0})(\bm{x^{*}}-\bm{x_0})=0
\end{equation}
其中$\bm{x^{*}}$是$f(\bm{x})$的极小值点,所以有:
\begin{equation}
    \bm{x^{*}}=\bm{x_0}-\nabla ^2 f(\bm{x_0})^{-1}\nabla f(\bm{x_0})
\end{equation}
将$\bm{x^{*}}$代入$f(\bm{x})$的二阶泰勒展开式,得到:
\begin{equation}
    \begin{aligned}
    f(\bm{x^{*}})&=f(\bm{x_0})-(\nabla^2f(\bm{x_0})^{-1}\nabla f(\bm{x_0}))^T\nabla f(\bm{x_0})\\
    &+\frac{1}{2}(\nabla^2f(\bm{x_0})^{-1}\nabla f(\bm{x_0}))^T\nabla^2f(\bm{x_0})(\nabla^2f(\bm{x_0})^{-1}\nabla f(\bm{x_0}))
    \end{aligned}
\end{equation}
因此我们得到牛顿减量的表达形式。$\lambda_f(x)$越小,说明当前点附近的二次模型逼近越可信,算法使用$\lambda_f$来判断是否终止迭代。在本节中,我们使用$\lambda_t=\lambda_f(x_t)$的记号来代表牛顿减量。
下面我们介绍在二次规划问题中,Sharpened-BFGS算法的更新结果,收敛速度(牛顿减量$\lambda_t$是如何收敛到零的)。

\begin{lema2}\label{lema:theone}
    对于形如式(1)的二次规划问题和步长为1的拟牛顿法产生的迭代序列,我们有:
    \begin{equation}
        \lambda_{t+1}=\theta(A,G_t,x_{t+1}-x_t)\lambda_t
    \end{equation}
    其中,
    \begin{equation}
        \theta(A,G,u) := \left(\frac{u^\top (G - A) A^{-1} (G - A) u}{u^\top G A^{-1} G u}\right)^{\frac{1}{2}}.
    \end{equation}
\end{lema2}

该引理通过收缩因子$\theta$量化了相邻两次迭代的牛顿减量之间的关系。观察收缩因子$\theta$的表达式,反映了矩阵$G$和$A$在非零方向$u$上的相似程度,用于评估迭代方向上Hessian矩阵的改进性能。
注意到连接两次牛顿减量的收缩因子表达式中的$x_{t+1}-x_t$项,它强调牛顿减量收敛的因子与$G_t(x_{t+1}-x_t),A(x_{t+1}-x_t)$之间的差距有关。

由于我们关心的是收敛速度问题,因此下面的定理给出Sharpened-BFGS算法下收缩因子的上界。

\begin{thrm2}
    使用Sharpened-BFGS算法求解二次规划问题,则:
    \begin{equation}
        \theta(A, G_t, x_{t + 1} - x_{t})\leq 1 - \frac{\mu}{L}, \qquad \forall t \geq 0,
    \end{equation}
    因此我们有:
    \begin{equation}
        \lambda_t \leq \left(1 - \frac{\mu}{L}\right)^{t}\lambda_0, \qquad \forall t \geq 0.
    \end{equation}
\end{thrm2}

该定理指出,在每次迭代过程中收缩因子的上界与问题参数$\mu,L$有关($\mu$强凸,$L$光滑),因此牛顿减量$\lambda_t$满足线性速率的衰减。
虽然能够以线性速率收敛,但是由于这个上界仅仅由$G_t,A$的特征值都有界的性质得出,因此这个估计可能比实际的上限更加宽松。在接下来的引理和定理中,我们可以看到收缩因子序列最终收敛于零,Sharpened-BFGS算法的收敛速度呈超线性。

\begin{lema2}
    考虑Sharpened-BFGS算法求解二次规划问题,进一步定义$\theta_t:=\theta(A, G_t, x_{t + 1} - x_{t})$,$\sigma_t := \sigma(A, G_t)$。则对于任意$t\ge 0$:
    \begin{equation}
        \sigma_{t+1} \leq \left(1 - \frac{\mu}{dL}\right)\left(\sigma_t - \theta_t^2\right)
    \end{equation}
    并且我们有:
    \begin{equation}
        \sum_{i = 0}^{t - 1}\frac{\theta^2_i}{(1 - \frac{\mu}{d L})^{i}}  \leq \sigma_0, \qquad \forall t \geq 1. 
    \end{equation}
\end{lema2}

回忆贪心BFGS算法中引入的函数$\sigma (A,G)$,它可以用来衡量矩阵$A,G$的距离(相似程度)。该引理证明了Sharpened-BFGS算法中的$\sigma_t$与贪心BFGS算法相比具有更快的向零收敛的速率。值得注意的是,引理也给出了收缩因子随迭代轮次增加的上界变化情况,整个$\theta_t$序列最终收敛到零。因此与前述定理相比,收缩因子具有更加紧密的上界。
由此我们可以得到牛顿减量超线性收敛的结论,由下面的定理给出。

\begin{thrm2}
    若使用Sharpened-BFGS算法求解二次规划问题,那么对于$t\ge 1$的情况:
    \begin{equation}
        \lambda_t \leq \left(1 - \frac{\mu}{dL}\right)^{\frac{t(t - 1)}{4}} \left(\frac{dL}{t\mu}\right)^{\frac{t}{2}}\lambda_0.
    \end{equation}
\end{thrm2}

观察牛顿减量收缩的表达式,其中$\left(1 - \frac{\mu}{dL}\right)^{\frac{t(t - 1)}{4}}$线性收敛,而$\left(\frac{dL}{t\mu}\right)^{\frac{t}{2}}$在$t>d\frac{L}{\mu}$时(即底数小于1时)超线性收敛。
因此,对于二次规划问题,Sharpened-BFGS算法在迭代次数小于$d\frac{L}{\mu}$时,以线性速度收敛,而在迭代次数大于$d\frac{L}{\mu}$时,以超线性速度收敛,$\lambda_t$以$\mathcal{O}((1-\frac{\mu}{dL})^{t^2} (\frac{dL}{\mu t})^t)$的速率收敛到零。


\subsection{一般的强凸光滑场景}

在本节中,我们将Sharpened-BFGS算法扩展到一般的情形。为了建立算法的超线性收敛速率,我们需要如下的两个假设。
\begin{assumption}
    目标函数$f$二阶可微。$f$是强凸函数,强凸参数为$\mu>0$, 并且$f$的梯度$\nabla f$是Lipschitz连续的,Lipschitz参数为$L>0$。
\end{assumption}
\begin{assumption}
    目标函数$f$是具有参数$M$的强自和谐函数。即对于任意$x,y,z,w\in \mathbb{R}^n$, 我们有:
    \begin{equation}
        \nabla^{2}{f(y)} - \nabla^{2}{f(x)} \preceq M\|y - x\|_{z}\nabla^{2}{f(w)}
    \end{equation}
    其中 $\|y - x\|_{z} : = \sqrt{(y - x)^\top \nabla^2{f(z)} (y - x)}$
\end{assumption}

假设二中引入的强自和谐函数是为了分析拟牛顿法的二次收敛速率。
观察它的表达式,$\nabla^{2}{f(y)} - \nabla^{2}{f(x)}$表示$f$在$y$和$x$处的Hessian矩阵的差异,即函数在这两点附近局部性质的变化。
$\|y - x\|_{z}$表示向量$y-x$通过$z$点Hessian矩阵的作用后产生变化的范数。该范数的引入确保了函数在点$x,y$之间的局部变化受到Hessian矩阵的适当控制。这个性质使得牛顿法在局部收敛时更为有效,可以证明算法在逼近最优解时的收敛速度至少是二次的。

下面给出一般的Sharpened-BFGS算法的伪代码:
\begin{algorithm}[H]\xiaosi
    \caption{\xiaosi General Sharpened-BFGS}
    \begin{algorithmic}{
        \REQUIRE 初始化变量$x_0$, 初始化近似矩阵$G_0=LI$
        \FOR{$t=0,1,2,\cdots$}
            \STATE 更新变量 $x_{t+1}=x_t-G_t^{-1}\nabla f(x_t)$;
            \STATE 计算BFGS方向 $s_t=x_{t+1}-x_t$;
            \STATE 计算沿$s_t$方向的平均海森矩阵,作为$G_t$需要近似的值 $J_t=\int_0^1\nabla^2f(x_t+\tau s_t)d\tau$;
            \STATE 计算 $\bar{G_t}=BFGS(J_t,G_t,s_t)$;
            \STATE 计算修正项 $r_t = \|x_{t + 1} - x_{t}\|_{x_t}$;
            \STATE 计算 $\hat{G_t} = (1 + {Mr_t}/{2})^2\bar{G_t}$;
            \STATE 计算贪心BFGS方向$\bar{u} = \bar{u}(\nabla^2{f(x_{t + 1})}, \hat{G_t})$;
            \STATE 估计海森矩阵 $G_{t + 1} = BFGS(\nabla^2{f(x_{t + 1})}, \hat{G_t}, \bar{u})$;
        \ENDFOR}
    \end{algorithmic}
\end{algorithm}
算法的整体流程和4.1节所述的二次规划问题类似,都是在每一轮迭代中进行两次BFGS的更新估计,其中第一次BFGS得到的矩阵用来贪婪计算下降方向。
与二次规划中的Sharpened-BFGS不同的地方在于,一般情形的算法在两次BFGS更新之间添加了修正项$r_t=\|x_{t + 1} - x_{t}\|_{x_t}$。
由于在一般情形中,函数的Hessian矩阵并不是固定的(二次规划中Hessian矩阵始终为$A$),我们并不能保证$\nabla^2 f(x)\preceq G$始终成立,而只有当$\nabla^2 f(x)\preceq G$时,$\sigma (\nabla^2 f(x),G)$才是良定义的。
因此我们需要添加修正项确保在经过一次BFGS更新之后,新的点$x_{+}$和新的Hessian逼近矩阵$G_{+}$仍满足$\nabla^2 f(x_{+})\preceq G_{+}$。
\begin{remark2}
    我们需要考虑算法的迭代计算成本是否因为修正项的增加而改变。注意到二次规划中的Sharpened-BFGS每轮迭代的计算成本为$\mathcal{O}(d^2)$。而计算修正向量$r_t$和修正矩阵$\hat{G_t}$的成本也是$\mathcal{O}(d^2)$,因此修正项的引入后算法每轮迭代的计算成本并没有发生变化。
\end{remark2}

对一般情形的Sharpened-BFGS算法收敛速率的分析大体上与二次规划的情形类似。但是正如上文所述,一般情形下目标函数的Hessian矩阵会发生变化,这是我们需要考虑的地方。
除此之外,需要注意的是在一般情形下,只有当初始点在最优解的局部邻域内,我们才能保证算法的收敛性。
和二次规划问题相似,接下来一步步给出一般情形下Sharpened-BFGS算法收敛速率的结论。首先我们给出牛顿减量的收缩因子。

\begin{lema2}
    考虑满足假设4.1和4.2的目标函数以及步长为1的拟牛顿法产生的迭代序列,我们有:
    \begin{equation}
        \lambda_{t+1} \leq \left(1 + \frac{Mr_t}{2}\right)\theta(J_t, G_t, x_{t + 1} - x_{t})\lambda_t,
    \end{equation}
    其中 $J_t := \int_{0}^{1}\nabla^2{f(x_t + \tau(x_{t + 1} - x_{t}))d\tau}$, $r_t := \|x_{t+1} - x_t\|_{x_t}$。
\end{lema2}

在给出了相邻两次迭代牛顿减量的关系之后,我们需要考虑收缩因子$\theta$的上界。与二次规划的步骤一致,下面的定理先给出较为宽松的上界。

\begin{thrm2}
    考虑一般情形下的Sharpened-BFGS算法,以及满足假设4.1和4.2的目标函数。假设初始点$x_0$满足:
    \begin{equation}
        \lambda_0\le\frac{C_0\mu}{ML}
    \end{equation}
    其中$C_0 = \frac{1}{4}\ln{\frac{3}{2}}$,则对任意$t\ge 0$,我们有:
    \begin{equation}
        \theta(J_t, G_t, x_{t + 1} - x_{t}) \leq 1 - \frac{2\mu}{3L},
    \end{equation}
    因此得出收缩因子的上界:
    \begin{equation}
        \lambda_t \leq (1 - \frac{\mu}{2L})^{t}\lambda_0.
    \end{equation}
\end{thrm2}

由于限制了初始点的牛顿减量,因此上述定理给出的是在最优解的局部邻域内,Sharpened-BFGS算法以$1-\frac{\mu}{2L}$的线性速率收敛。接下来的引理和定理将给出收缩因子更精确的上界,使得算法在一般情形下也能达到超线性的收敛速率。

\begin{lema2}
    考虑一般情形下的Sharpened-BFGS算法,以及满足假设4.1和4.2的目标函数。假设初始点$x_0$满足:
    \begin{equation}
        \lambda_0\le\frac{C_0\mu}{ML}
    \end{equation}
    其中$C_0 = \frac{1}{4}\ln{\frac{3}{2}}$。定义$\theta_t := \theta(\nabla^2{f(x_{t})}, G_t, x_{t + 1} - x_{t})$,$\sigma_t := \sigma(\nabla^2{f(x_{t})}, G_{t})$.则对任意$t\ge 0$,我们有:
    \begin{equation}
        \sigma_{t+1} \leq (1 - \frac{\mu}{2dL})\left[(1 + \frac{M\lambda_t}{2})^4(\sigma_t + 4Md\lambda_t) - \frac{1}{4}\theta^2_t\right].
    \end{equation}
    以及:
    \begin{equation}
        \sum_{i = 0}^{t - 1}\frac{\theta^2_i}{(1 - \frac{\mu}{2d L})^{i}}  \leq 8(\sigma_0 + 4Md\lambda_0), \qquad \forall t \geq 1. 
    \end{equation}
\end{lema2}

上述引理也是建立在最优解的局部邻域之内。具体而言,式(20)说明只要$\lambda_0$足够小(初始点位于最优解的局部邻域),那么$\sigma_t$可以收敛到零,即$G_t$能够很好地近似实际的Hessian矩阵$\nabla^2 f(x_t)$,并且由于$\theta_t^2$的存在,其收敛速率快于贪心BFGS算法。
式(21)给出收缩因子$\theta$更精确的上界,保证其能够以更快的速度收敛到零。有了上述的结论,我们可以得出如下的定理,给出一般情形下Sharpened-BFGS算法的收敛速率。

\begin{thrm2}
    考虑一般情形下的Sharpened-BFGS算法,以及满足假设4.1和4.2的目标函数。假设初始点$x_0$满足:
    \begin{equation}
        \lambda_0 \leq \frac{C_1\mu}{dML},
    \end{equation}
    其中$ C_1 = \frac{\ln{2}}{20}$。那么对于任意的$t\ge 1$,我们有:
    \begin{equation}
        \lambda_t \leq 2\left(1 - \frac{\mu}{2dL}\right)^{\frac{t(t - 1)}{4}} \left(\frac{8dL}{t\mu}\right)^{\frac{t}{2}}\lambda_0.
    \end{equation}
\end{thrm2}

类似于二次规划问题的分析,第一项$\left(1 - \frac{\mu}{2dL}\right)^{\frac{t(t - 1)}{4}}$线性收敛,第二项$\left(\frac{8dL}{t\mu}\right)^{\frac{t}{2}}$在$t\ge\frac{8dL}{t\mu}$时超线性收敛。
总结来说,当迭代轮数$t \leq \Theta(d\frac{L}{\mu})$时,算法具有线性收敛速率,当迭代轮数$t \geq \Theta(d\frac{L}{\mu})$时,算法可以达到超线性的收敛速率。
至此我们介绍了Sharpened-BFGS算法的具体思路方法,以及收敛性分析的推导过程。

%-----------------------
%
%-----------------------
\section{理论结果}\label{section:theory}
% 对应原论文的 Discussions 一节,在理论上比较一下三种算法的收敛速度,以及开始超线性收敛所需的步数
这一部分我们将Sharpened-BFGS的收敛结果与第三部分的经典BFGS和贪心BFGS进行比较。我们特别关注目标函数满足假设4.1和4.2的情况。为了简化比较,除了参数$\mu,L,M,d$之外,其它的参数我们用常数$1$来替代,并且定义条件数$\kappa = {L}/{\mu} \geq 1$。

\noindent \textbf{Sharpened-BFGS.}
根据第四部分的结果,如果我们设置初始近似矩阵$G_0=LI$,并且初始点$x_0$满足:
\begin{equation*}
    \lambda_f(x_0) = \mathcal{O}\left(\frac{1}{dM\kappa}\right),
\end{equation*}
那么由Sharpened-BFGS算法产生的迭代序列满足:
\begin{equation*}
    \frac{\lambda_f(x_t)}{\lambda_f(x_0)} \leq
    \min\left\{\left(1 \!-\! \frac{1}{\kappa}\right)^t ,\  \left(1 \!-\! \frac{1}{d\kappa}\right)^{\!\frac{t(t - 1)}{4}}\!\! \left(\frac{d\kappa}{t}\right)^{\frac{t}{2}} \right\}.
\end{equation*}
从表达式可以看出,当$t<d\kappa$时,第一个上界项更小,牛顿减量以线性速率$\left(1 \!-\! \frac{1}{\kappa}\right)^t$收敛。当$t>d\kappa$时,第二个上界项更小,牛顿减量以$\left(1 \!-\! \frac{1}{d\kappa}\right)^{\!\frac{t(t - 1)}{4}}\!\! \left(\frac{d\kappa}{t}\right)^{\frac{t}{2}}$的超线性速率收敛,并且可以看出其收敛速度快于二次收敛。

\noindent \textbf{Greedy-BFGS.}
如果我们设置初始近似矩阵$G_0=LI$,并且初始点$x_0$满足:
\begin{equation*}
    \lambda_f(x_0) = \mathcal{O}\left(\frac{1}{dM\kappa}\right),
\end{equation*}
那么由Greedy-BFGS算法产生的迭代序列满足:
\begin{equation*}
    \frac{\lambda_f(x_t)}{\lambda_f(x_0)} \leq
    \min\left\{\left(1 \!-\! \frac{1}{\kappa}\right)^t ,\  \left(1 \!-\! \frac{1}{d\kappa}\right)^{\!\frac{t(t - 1)}{2}}\!\! \left(\frac{1}{2}\right)^{t} \right\}.
\end{equation*}
将其结果与Sharpened-BFGS的结果进行对比,我们可以发现:

\begin{enumerate}
    \item Greedy-BFGS算法的迭代轮数达到$d\kappa\ln{(d\kappa)}$之后,收敛速度达到超线性。这是慢于Sharpened-BFGS算法($d\kappa$轮后达到超线性)的。
    \item 由于当$t$充分大时,除了共同的二次收敛项$(1 - \frac{1}{d\kappa})^{t^2}$之外,$(\frac{d\kappa}{t})^{\frac{t}{2}} \ll (\frac{1}{2})^{t}$,因此Sharpened-BFGS算法最终的超线性收敛速率快于Greedy-BFGS算法。
\end{enumerate}

\noindent \textbf{BFGS.}
如果我们设置初始近似矩阵$G_0=LI$,并且初始点$x_0$满足:
\begin{equation*}
    \lambda_f(x_0) = \max\left\{\mathcal{O}\left(\frac{1}{M\kappa}\right), \mathcal{O}\left(\frac{1}{Md\ln{\kappa}}\right)\right\},
\end{equation*}
那么由BFGS算法产生的迭代序列满足:
\begin{equation*}
    \frac{\lambda_f(x_t)}{\lambda_f(x_0)} \leq
    \min\left\{\left(1 \!-\! \frac{1}{\kappa}\right)^t ,\ \left(\frac{d\ln{\kappa}}{t}\right)^{\frac{t}{2}} \right\}.
\end{equation*}

我们可以看到,BFGS算法在经过$d\ln\kappa$轮后达到超线性收敛速率,这是快于Sharpened-BFGS算法的。
但是,当$t$充分大时我们有
\begin{equation*}
    \left(1 \!-\! \frac{1}{d\kappa}\right)^{\!\frac{t(t - 1)}{4}}\!\! \left(\frac{d\kappa}{t}\right)^{\frac{t}{2}}
    \ll 
    \left(\frac{d\ln{\kappa}}{t}\right)^{\frac{t}{2}}.
\end{equation*}
因此BFGS算法的超线性收敛速率是慢于Sharpened-BFGS算法的。

%-----------------------
%
%-----------------------
\section{实验结果}\label{section:experiment}
% 对应原论文的 Numerical Experiments 一节,对带正则化项的逻辑斯蒂回归函数做实验
% 我觉得应该是要自己实现算法做实验,然后把实验结果画图放进来

%-----------------------
%
%-----------------------
\section{问题分析与挑战}\label{section:problem}
% 做了可以酌情加分,要不要做一下?
% 可以就目标函数的局限性展开分析,比如说假设过于严苛,对于一般的问题不适用;实验只做了逻辑斯蒂回归,其他实验可能会出现 bad case(可以构造别的函数也做一下实验?)

%-----------------------
%
%-----------------------
\section{总结}\label{section:conclusion}
% 对应原论文的 Conclusions 一节,翻译一下原论文的总结,再加上一点自己的感悟,应该比较简单
在本篇论文中,作者提出了一种用于解决无约束凸优化问题新的拟牛顿方法——Sharpened-BFGS。其中目标函数具有$\mu$-强凸性,其梯度具有$L$-光滑性,并且是$M$-强自和谐的。
Sharpened-BFGS算法充分利用了经典BFGS算法在牛顿方向的近似和贪心BFGS算法的Heissan矩阵近似。利用这些性质,作者证明了该算法达到$\mathcal{O}((1 - \frac{\mu}{dL})^{\frac{t(t - 1)}{4}} (\frac{dL}{t\mu})^{\frac{t}{2}})$的超线性收敛速率,并且其收敛速度快于二次收敛。
作者也通过理论分析和数值实验将该方法和经典BFGS和Greedy-BFGS算法的收敛速率进行了比较,凸显了Sharpened-BFGS算法的优越性。

在学习这篇论文的工作时,我们体会到了不同优化方法的差异以及如何充分利用它们各自的优势得到表现更好的优化方法。同时,我们在跟着论文一步步推导并最终得到结果的过程中也学习到了如何从理论上分析算法的收敛速度,以及如何通过数值实验来验证理论结果,受益匪浅。

%+++++++++++++++++++++++++++++++++++++++++++++++++++++++++++++++++++++++++++++
% Bibliography
%+++++++++++++++++++++++++++++++++++++++++++++++++++++++++++++++++++++++++++++
\bibliographystyle{gbt7714-plain}
\bibliography{main}

%+++++++++++++++++++++++++++++++++++++++++++++++++++++++++++++++++++++++++++++
\end{document}
%+++++++++++++++++++++++++++++++++++++++++++++++++++++++++++++++++++++++++++++


