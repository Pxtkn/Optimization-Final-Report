% !Mode:: "TeX:UTF-8"% !TEX TS-program = xelatex
% !TEX encoding = UTF-8 Unicode
% !Mode:: "TeX:UTF-8"

%+++++++++++++++++++++++++++++++++++++++++++++++++++++++++++++++++++++++++++++
% This is a sample article script. All rights reserved.
% Author: qianhui@zju.edu.cn
%+++++++++++++++++++++++++++++++++++++++++++++++++++++++++++++++++++++++++++++
\documentclass[a4paper,twoside,AutoFakeBold]{article}
\usepackage{optreport}

%+++++++++++++++++++++++++++++++++++++++++++++++++++++++++++++++++++++++++++++
% Some packages for this sample.
%+++++++++++++++++++++++++++++++++++++++++++++++++++++++++++++++++++++++++++++
\usepackage{comment}	% Package for comment useless document
\usepackage{bm}			% Package for Bold-math symbol
\usepackage{mathrsfs}	% Package for RSFS fonts in maths
\usepackage{listings}	% Package for Listing code
\usepackage{enumerate}	% Package for enumerate
\usepackage{pdfrender}
\usepackage{subcaption}
\usepackage{multirow}
\usepackage{booktabs}
\usepackage{bbm}
\usepackage{bbding}
\usepackage{mathtools}
\usepackage{graphicx}
\usepackage{float}
\usepackage{epstopdf}
\usepackage{lipsum}
\usepackage{metalogo}

%+++++++++++++++++++++++++++++++++++++++++++++++++++++++++++++++++++++++++++++
% Title, Authors, Reprot Time.
%+++++++++++++++++++++++++++++++++++++++++++++++++++++++++++++++++++++++++++++
\serialnum{2024-3-1145141919}

\rptname{Optimal Transport Based Distributed \gaplongcap Optimization Research}

\rptauthora{郑皓壬}{3220103230} %作者1和学号
\rptauthorb{郑俊达}{3220103540} %作者2和学号
\rptauthorc{李瀚轩}{3220106039} %作者3和学号
\reporttime{2024}{1}

% -------------------------------------------------
% for english version.
% -------------------------------------------------
\rptcontentsname{Contents}
\renewcommand{\abstractname}{{\xiaosan Abstract}}
\def\bibetal{et al.}
\def\biband{and}
\makeatletter
\renewcommand*{\ALG@name}{{\xiaosi Algorithm.~}}
\makeatother
\theoremstyle{definition}
\newtheorem{defn2}{{Definition}}
\newtheorem{corr2}{{Corrollary}}
\newtheorem{thrm2}{{Theorem}}
\newtheorem{lema2}{{Lemma}}
\newtheorem{exmp2}{{Example}}
\newtheorem{remark2}{{Remark}}
\renewcommand*{\proofname}{{\heiti Proof.~}}
\renewcommand{\figurename}{Fig.~}
\renewcommand{\tablename}{Tab.~}
\renewcommand{\refname}{Reference}
% -------------------------------------------------

%+++++++++++++++++++++++++++++++++++++++++++++++++++++++++++++++++++++++++++++
% Document.
%+++++++++++++++++++++++++++++++++++++++++++++++++++++++++++++++++++++++++++++
\begin{document}
\pagenumbering{gobble}

%-----------------------------------------------------------------------------
%  Title Page
%-----------------------------------------------------------------------------
\maketitle
\thispagestyle{empty} \cleardoublepage

%-----------------------------------------------------------------------------
%  Table of Content
%-----------------------------------------------------------------------------
\rptcontent \thispagestyle{empty} \cleardoublepage

%-----------------------------------------------------------------------------
%  Abstract
%-----------------------------------------------------------------------------
\begin{abstract}\kaiti \xiaosi
% 论文的摘要,大致翻译一下原论文的摘要即可
\end{abstract}
\cleardoublepage

%-----------------------------------------------------------------------------
%  Sections
%-----------------------------------------------------------------------------
\pagenumbering{arabic}\songti\xiaosi
%-----------------------
%
%-----------------------
\section{论文介绍}
\subsection{问题背景}
% 论文的问题背景,对应原论文 Introduction 中 Conrtribution 之前的部分,应该比较简单
\cite{zhangzhihua2021quasinewton1}

\subsection{论文贡献}
% 论文的贡献,对应原论文 Introduction 中的 Conrtribution 部分,应该比较简单

\subsection{章节组织}
% 论文的章节组织,无对应部分,大概是要简单介绍一下原论文的章节组织?

%-----------------------
%
%-----------------------
\section{相关工作}\label{section:related}
% 论文的相关工作,对应原论文 Introduction 中的 Related Work 部分,应该比较简单

%-----------------------
%
%-----------------------
\section{问题描述和常用记号}\label{section:preliminary}
% 对应原论文的 Preliminary 一节,需要介绍经典 BFGS 和 Greedy-BFGS 算法
% 里面的引理的证明应该不用写吧?可以加一句“证明请详见原论文附录”?

\subsection{BFGS算子与算法}

\subsection{Greedy-BFGS算法}

%-----------------------
%
%-----------------------
\section{方法描述}\label{section:methods}
% 对应原论文的 Sharpened-BFGS 一节,需要介绍 Sharpened-BFGS 算法的思路及其用到的各种引理、定理
% 应该也不用证明吧?不过加上证明可以水不少分(x)
% 我觉得原论文中有一些没详细说的部分可以扩展一下,比如关于 Newton's Decrement 的定义,可以写一下 $\lambda$ 的公式是怎么来的

\subsection{二次规划}

\subsection{一般的强凸光滑场景}


%-----------------------
%
%-----------------------
\section{理论结果}\label{section:theory}
% 对应原论文的 Discussions 一节,在理论上比较一下三种算法的收敛速度,以及开始超线性收敛所需的步数

%-----------------------
%
%-----------------------
\section{实验结果}\label{section:experiment}
% 对应原论文的 Numerical Experiments 一节,对带正则化项的逻辑斯蒂回归函数做实验
% 我觉得应该是要自己实现算法做实验,然后把实验结果画图放进来

%-----------------------
%
%-----------------------
\section{问题分析与挑战}\label{section:problem}
% 做了可以酌情加分,要不要做一下?
% 可以就目标函数的局限性展开分析,比如说假设过于严苛,对于一般的问题不适用;实验只做了逻辑斯蒂回归,其他实验可能会出现 bad case(可以构造别的函数也做一下实验?)

%-----------------------
%
%-----------------------
\section{总结}\label{section:conclusion}
% 对应原论文的 Conclusions 一节,翻译一下原论文的总结,再加上一点自己的感悟,应该比较简单



%+++++++++++++++++++++++++++++++++++++++++++++++++++++++++++++++++++++++++++++
% Bibliography
%+++++++++++++++++++++++++++++++++++++++++++++++++++++++++++++++++++++++++++++
\bibliographystyle{gbt7714-plain}
\bibliography{main}

%+++++++++++++++++++++++++++++++++++++++++++++++++++++++++++++++++++++++++++++
\end{document}
%+++++++++++++++++++++++++++++++++++++++++++++++++++++++++++++++++++++++++++++


